\documentclass[]{thesis-ekf}
\usepackage[T1]{fontenc}
\PassOptionsToPackage{defaults=hu-min}{magyar.ldf}
\usepackage[magyar]{babel}
\usepackage{mathtools,amssymb,amsthm}
\footnotestyle{rule=fourth}

\begin{document}
\institute{Matematikai és Informatikai Intézet}
\title{Játékfejlesztés Unity keretrendszerben}
\author{Kurta Gábor Levente\\Programtervező Informatikus Bsc}
\supervisor{Troll Ede \\tanár segéd}
\city{Eger}
\date{2020}
\maketitle
\tableofcontents

\chapter{Bevezetés}
\section{Miért a játékfejlesztés?}
A mai napig élesen megmaradt bennem az a mondat, amit az egyik tanárunk mondott nem is olyan régen. Volt egy gyakorlati óránk 1 évvel ezelőtt, ami annyiból állt, hogy minden tanár bejött tartani egy kisebb előadást, melyen elmondta milyen témákat lehet nála választani szakdolgozat tekintetében, illetve megosztotta tapasztalatait általánosságban a szakdolgozat írásáról. Az egyik tanár azt mondta egyszer, hogy nagyon óvatosan és előregondolkodva válasszunk témát, ugyanis aki olyan témában szeretné megírni a dolgozatát amit még nem ismer, az könnyen rengeteg fennakadásba ütközhet a technológia megismerése közben. Azért szúrt nekem szemet ez a gondolat, mert én pont ezt akartam. Eldöntöttem még régebben, hogy én játékot szeretnék fejleszteni a tanulmányaim végén, de sajnálatos módon soha nem hallottam egy játékfejlesztő keretrendszerről sem. Semmit nem tudtam erről az egészről, mit kell letöltenem hogy nekiláthassak, mi a menete a fejlesztésnek, mekkora előretervezést igényel, de én még is ezzel akartam foglalkozni. Az elején egy kicsit megingott a bizalmam magamban, hogy őszinte legyek. Féltem attól, hogy a technológia megismerése és elsajátítása olyan sok időt fog felemészteni, hogy nem marad időm a tényleges szakdolgozat megvalósítására, és csak a kapkodás lesz belőle, illetve egy rossz jegy, és egy összecsapott munka. Amikor leadtam a jelentkezésemet a megfelelő tanárhoz, tudatosult bennem, hogy ennek bizony hamar neki kell látni, ugyanis jelenleg azt sem tudom mi fán terem a játékfejlesztés.

\section{Előkészületek}
Amikor először beszéltem a konzulensemmel a szakdolgozattal kapcsolatban, egy kicsit megnyugodtam. Azt mondta, nem olyan nehéz megbarátkozni a technológiával, és tudott nekem adni ő általa megvett kurzusokat, melyek lépésről lépésre végig vezetnek egy-egy adott játék fejlesztésén. Van egy oktató, aki videókat készít arról, hogy hogyan programozza le a játékot. Minden apró részlet külön videóban van taglalva, és mindent alaposan elmagyaráz, hogy mindenki megértse. A játékok fejlesztése döntő többségben C\# nyelven történt, ami az előnyömre vált, mivel ismertem a nyelvet, hiszen ezt tanultuk első félévtől kezdve.
\subsection{2D kurzus}
Az első kurzus amit végigcsináltam egy nagyon egyszerű, űrhajós lövöldözős játék volt, amelyben a játékos egy űrhajót irányított és ellenfelek jelentek meg a pályán, amiket le kellett lőnie, vagy kikerülnie. Ez a játék arra volt jó, hogy megismerkedjek a Unity környezetével, komfortosan érezzem magam a használata közben és megtanuljam az alapokat.
\chapter{A játékfejlesztésről általánosságban}
\section{Technika történet}


\section{Játékfejlesztésben használt motorok}
Itt fogok írni arról, hogy a játékfejlesztésben milyen motorokat használnak manapság, illetve használtak régebben.
\subsection{Unity}
Unity.
\subsection{Unreal Engine}
\subsection{Motorok összehasonlítása}
Itt fogom összehasonlítani a két főbb játékfejlesztési motort, a Unityt és az Unreal Enginet.

\chapter{Verziókövetés}
\section{Előnyök}
Itt fogok általánosságban írni a verziókövető rendszerekről, hogy miért hasznosak.
\section{Git}
Itt majd az általam használt verziókövető rendszerről (git) fogok írni.

\chapter{Saját játék}
\section{Bemutatás}
Itt befogom mutatni azt a játékot, amit elkészítettem. Milyen a pálya, mi volt az alap ötlet.
\subsection{Felhasználói interakció}
c betű karakterlap, jobb klikk kamera mozgatás, wasd movement stb.
\section{Karakter}
Itt fogok a karakterről írni.
\subsection{Mozgás}
Karakter mozgása
\subsection{Kamera}
Kamera mozgatás, követi a playert stb.
\section{Ellenfelek}
Itt az ellenfelekről fogok beszélni.
\subsection{Enemy}
Sima ácsorgó ellenfél.
\subsection{PatrollingEnemy}
Járkáló ellenfél.
\subsection{BOSS}
BOSS.
\section{Inventory rendszer}
Itt fogok írni az inventory rendszerről.
\section{Quest rendszer}
Itt fogok írni a quest rendszerről.
\section{Szint rendszer}
Itt a szint rendszerről fogok írni.
\section{Gyűjthető és felhasználható tárgyak}
Itt fogok írni a hp potiról és a healről.

\chapter{Ismert hibák és továbbfejlesztési lehetőségek}
\section{Ismert hibák}
Itt fogok írni a játékban ismert hibákról.
\section{Továbbfejlesztés}
Itt fogok írni a továbbfejlesztési lehetőségekről. Quest rendszer pl lehetne máshogyan megoldva, adatbázisban tárolva, enemy ne rotáljon el függőlegesen ha közelmegy a player stb.

\chapter{Felhasznált assetek}
Itt fogok írni az általam felhasznált assetekről,lehet hogy külön section-be szedem a questhez, karakterhez, droppolt itemekhez, pályához használt asseteket.

\chapter{Ábra jegyzék}
Itt az internetről letöltött és saját készítésű ábrák jegyzéke lesz. (mi is ez pontosan???)


\begin{thebibliography}{2}
\end{thebibliography}
\end{document}